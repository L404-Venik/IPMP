\documentclass{article}

\usepackage[utf8x]{inputenc}
\usepackage[english,russian]{babel}
\usepackage{graphicx}
\usepackage{amsmath}
\usepackage{amssymb}
\usepackage{extarrows}
\usepackage{vmargin}
\usepackage{MnSymbol}
\setpapersize{A4}
\setmarginsrb{2cm}{2cm}{2cm}{2cm}{0pt}{0mm}{0pt}{13mm}
%\usepackage{cmap}

\newtheorem{theorem}{Теорема}


\begin{document}

\centerline{\large Курс лекция для магистров ВМК МГУ}
\centerline {\textbf{\LARGE Обратные задачи математической физики}}
\centerline {Затехал Строков Вениамин 2025}

\vspace{0.4cm}

\centerline{\LARGE 	Лекция 5. Некорректные и обратные задачи распространения волн}
\centerline{\LARGE 	в трещеноватой среде}

\vspace{0.5cm}

\centerline{\large Волновое уравнение с комплексной скоростью}

На трещинах происходит рэлеевское рассеяние
\[
\begin{cases}
s_{tt} = a^2 s_{xx} - 2 \nu p_{tt}, & 0 < \nu \leqslant 1;\\
p_{tt} = a^2 p_{xx} + 2 \nu s_{tt}, & s ~ \sqrt{\nu}
\end{cases}
\]

$\{p,s\} = \overrightarrow{w} = p + i s$, $ \overrightarrow{w}_{tt} = a^2(1- i\nu)^2 \overrightarrow{w}_{xx}$ (с точностью до $\nu^2$)

Рассмотрим частное решение вида 
$\overrightarrow{w} = \overrightarrow{A}(k) e^{i(\omega t - k x)}$. 
Возникает условие разрешимости системы

$$
\begin{matrix}
\omega^2 - a^2 k^2  	&  2 \nu \omega^2 \\
- 2 \nu \omega^2		&  \omega^2 - a^2 k^2
\end{matrix}
$$


\end{document}