\documentclass{article}

\usepackage[utf8x]{inputenc}
\usepackage[english,russian]{babel}
\usepackage{graphicx}
\usepackage{amsmath}
\usepackage{amssymb}
\usepackage{extarrows}
\usepackage{vmargin}
\usepackage{MnSymbol}
\setpapersize{A4}
\setmarginsrb{2cm}{2cm}{2cm}{2cm}{0pt}{0mm}{0pt}{13mm}
%\usepackage{cmap}

\newtheorem{theorem}{Теорема}


\begin{document}

\centerline{\large Курс лекция для магистров ВМК МГУ}
\centerline {\textbf{\LARGE Обратные задачи математической физики}}
\centerline {Затехал Строков Вениамин 2025}

\vspace{0.4cm}

\centerline{\LARGE 	Лекция 6. Обратная задача для волнового уравнения на отрезке}

\vspace{0.5cm}

Было доказано, что уравнение колебаний неоднородной струны
\begin{equation}
\rho(y) \omega_{tt} = \hat{T}_0 \omega_{yy}, \hspace{0.5cm} \rho \in C^1;
\end{equation}

можно привести к виду ($u = u(x,y)$)
\begin{equation}
u_{tt} = u_{xx} - q(x) u.
\end{equation}

\begin{equation*}
q(x) = z'(x) + z^2(x), \hspace{0.5cm} z = -\dfrac{c'(x)}{2 c(x)},
\end{equation*}

\begin{equation*}
x(y) = \int_0^y \dfrac{d \eta}{a(\eta)}, \hspace{0.5cm} a(y) = \sqrt{\hat{T}_0/ \rho(y)},
\end{equation*}

\begin{equation}
u = \omega(y(x),t)/ \sqrt{c(x)}, 
\end{equation}

$y(x)$ - функция, обратная к эйконалу $x(y)$. $c(x)$ - скорость в переменной эйконал. $z(x)$ имеет физический смысл коэффициента отражения.

Сделаем упрощающие замены
\begin{equation*}
a(y) = \sqrt{\hat{T}_0/ \rho(y)}, \hspace{0.5cm} a_L^2 = \hat{T}_0 / \rho_0
\end{equation*}

Введём новую независимую переменную - эйконал
\begin{equation*}
 x(y) = \int_0^y \dfrac{d \eta}{a(\eta)}
\end{equation*}
 - время распространения сигнала от $y$ до 0.
 
\begin{equation}
\hat{\omega}_{tt} = c(x) \dfrac{\partial}{\partial x} \left(\dfrac{1}{c(x)} \dfrac{\partial}{\partial x} \omega \right) = \hat{\omega}_{xx} - \dfrac{c'(x)}{c(x)} \hat{\omega}_x,
\end{equation}


\end{document}