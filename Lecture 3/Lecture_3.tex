\documentclass{article}

\usepackage[utf8x]{inputenc}
\usepackage[english,russian]{babel}
\usepackage{graphicx}
\usepackage{amsmath}
\usepackage{amssymb}
\usepackage{extarrows}
\usepackage{vmargin}
\usepackage{MnSymbol}
\setpapersize{A4}
\setmarginsrb{2cm}{2cm}{2cm}{2cm}{0pt}{0mm}{0pt}{13mm}
%\usepackage{cmap}

\newtheorem{theorem}{Теорема}


\begin{document}

\centerline{\large Курс лекция для магистров ВМК МГУ}
\centerline {\textbf{\LARGE Обратные задачи математической физики}}
\centerline {Затехал Строков Вениамин 2025}

\vspace{0.4cm}

\centerline{\LARGE Лекция 3. Задача определение формы}

\vspace{1cm}
\centerline{\large Единственность}

Рассмотрим класс тел, имеющих средние плоскости, т.е. для них $\exists$ плоскость $P$ такая, что если $Oz \bot P$, то поверхность тела описывается уравнением
\[
z = \varphi(x,y), z = \psi(x,y).
\]

\begin{theorem}
(единственности): \\
При постоянной известной плотности двух тел допустим, что тела обладают параллельными средними плоскостями и центры тяжести расположены внутри тел. Тогда при равенстве внешних потенциалов тела совпадают.
\end{theorem}
Доказательство:

Если
\[
v(M) = \iiint_T \dfrac{\sigma(M')}{z_{MM'}} d \tau_{M'} = 0,
\]
вне тела $T$, то $\sigma \bot \forall$ гармонической функции $u(M)$ в $T$. Действительно, из II формулы Грина имеем:
\[
\iiint_T (\Delta v u - \Delta u v) d \tau = \{\Delta u = 0\} = \iint_{\partial T} [\dfrac{\partial v}{\partial n} u - \dfrac{\partial u}{\partial n} v] ds = 0 = - 4 \pi \iiint \sigma u d\tau.
\]

Рассмотрим тело $T_1 \bigcup T_2$, для $T_1$, $T_2$, имеющих параллельные средние плоскости $z = z_1$ и $z = z_2$.
Тогда $\forall$ гармонической функции $u(M)$ в $T$:
\[
\iiint_{T = T_1 \bigcup T_2} \sigma (M) u(M) d\tau_M = 0 
\]
где
\[
\sigma(M) = 
	\begin{cases}
	1, & M \in T_1 \setminus (T_1 \bigcap T_2);\\
	0, & M \in T_1 \bigcap T_2;\\
	-1, &M \in T_2 \setminus (T_1 \bigcap T_2).
	\end{cases}
\]
т.е. $\sigma (M) = \sigma_1 - \sigma_2$ - разность плотностей.

\vspace{0.5cm}
\includegraphics[scale=0.85]{pic1.png}

Рассмотрим 
\[
f(P) = 
	\begin{cases}
	1, & z < z_1, P \in S_1;\\
	1, & z > z_2, P \in S_2;\\
	0, & \text{иначе}.
	\end{cases}
\]
и поставим задачу Дирихле в $T$:
\[
\begin{cases}
\Delta F = 0, & M \in T;\\
F \bigg|_S = f, & S = \partial T.
\end{cases}
\]







\end{document}