\documentclass{article}

\usepackage[utf8x]{inputenc}
\usepackage[english,russian]{babel}
\usepackage{graphicx}
\usepackage{amsmath}
\usepackage{amssymb}
\usepackage{extarrows}
\usepackage{vmargin}
\usepackage{MnSymbol}
\setpapersize{A4}
\setmarginsrb{2cm}{2cm}{2cm}{2cm}{0pt}{0mm}{0pt}{13mm}
%\usepackage{cmap}


\begin{document}

\centerline{\large Курс лекция для магистров ВМК МГУ}
\centerline {\textbf{\LARGE Обратные задачи математической физики}}
\centerline {Затехал Строков Вениамин 2025}

\vspace{0.4cm}

\centerline{\LARGE Лекция 3. Задача определение формы}

\vspace{1cm}
\centerline{\large Единственность}

Рассмотрим класс тел, имеющих средние плоскости, т.е. для них $\exists$ плоскость $P$ такая, что если $Oz \bot P$, то поверхность тела описывается уравнением
\[
z = \varphi(x,y), z = \psi(x,y).
\]

Теорема (единственности): 



\end{document}