\documentclass{article}

\usepackage[utf8x]{inputenc}
\usepackage[english,russian]{babel}
\usepackage{graphicx}
\usepackage{amsmath}
\usepackage{amssymb}
\usepackage{extarrows}
\usepackage{vmargin}
\usepackage{MnSymbol}
\setpapersize{A4}
\setmarginsrb{2cm}{2cm}{2cm}{2cm}{0pt}{0mm}{0pt}{13mm}

%user commands
\newtheorem{theorem}{Теорема}
\newtheorem{lemma}{Лемма}

\newenvironment{proof}{\paragraph{Доказательство:}}{\hfill$\blacksquare$}
\newenvironment{definition}{ \paragraph{Определение:}}{\hfill $\blacktriangleleft$}
\newenvironment{observation}{ \paragraph{Замечание:}}{\\}
\newenvironment{hence}{ \paragraph{Следствие:}}{}


\begin{document}

\centerline{\large Курс лекция для магистров ВМК МГУ}
\centerline {\textbf{\LARGE Обратные задачи математической физики}}
\centerline {Затехал Строков Вениамин 2025}

\vspace{0.4cm}

\centerline{\LARGE 	Лекция 11. Обратная задача электроразведки и }
\centerline{\LARGE магнитотелургического зондирования.}

\vspace{0.5cm}

Принципиальная схема электроразведки на постоянном токе:

%\includegraphics[scale=1]{pic11_1.png}

\begin{equation}
\begin{cases}
	\overrightarrow{j} = \sigma \overrightarrow{E};\\
	\text{div} \overrightarrow{j} = 0; \quad (\text{нет источников поля})\\
	\overrightarrow{E} = - \text{grad } v, \quad \Rightarrow \quad rot \overrightarrow{E} = 0. 
\end{cases}
\end{equation}

где $\sigma$ - проводимость. 

Далее считаем $\sigma = \sigma(z) \geqslant \sigma_0 > 0$ источник тока в начале координат
\begin{equation*}
	\text{div } \sigma \text{ grad } v = 0, \quad z > 0.
\end{equation*}

и в силу аксиальной симметрии уравнение
\begin{equation*}
	\dfrac{\partial^2}{\partial x^2} + \dfrac{\partial^2 v}{\partial y^2} + \dfrac{1}{\sigma(z)}\dfrac{\partial}{\partial z} ( \sigma(z) \dfrac{\partial v}{\partial z}) = 0, \quad z > 0
\end{equation*}

приобретает вид
\begin{equation*}
	\dfrac{1}{\rho} \dfrac{\partial}{\partial \rho} \left( \rho \dfrac{\partial v}{\partial \rho} \right) +
	 \dfrac{1}{\sigma(z)} \dfrac{\partial}{\partial z} \left( \sigma(z) \dfrac{\partial v}{\partial z} \right) = 0
	 \quad z > 0,
\end{equation*}

а решение имеет вид $v(\rho, z) = \dfrac{q}{\sqrt{\rho^2 + z^2}} + \overline{v}(\rho,z)$, $q > 0$, $v(\rho,z) \rightarrow 0$ при $z \rightarrow 0$.
$\dfrac{\partial v}{\partial z} \bigg|_{z=0} = 0$.

Будем искать $v(\rho,z)$ в виде интеграла Фурье-Бесселя:
\begin{equation}
	v(\rho,z) = \int_0^{\infty} Z(z,\lambda) J_0(\lambda\rho) \lambda d \lambda,
\end{equation}

где $|Z(z,\lambda)| < c e^{-\lambda z}$, $\lambda > 0$.
Найдём такую $Z(z,\lambda)$.

Напомним что
\begin{equation*}
	\dfrac{1}{\rho} \dfrac{\partial}{\partial \rho} \left( \rho \dfrac{\partial}{\partial \rho} J_0(\lambda \rho) \right) + \lambda^2 J_0(\lambda \rho) = 0.
\end{equation*}

Поскольку интеграл сходится равномерно по $\rho, z$ ( $z$ фиксировано, $\rho$ внешняя переменная), то:
\begin{equation*}
	\dfrac{1}{\sigma(z)}\dfrac{\partial}{\partial z} \left( \sigma(z) \dfrac{\partial v}{\partial z} \right) =
	- \dfrac{1}{\rho} \dfrac{\partial}{\partial \rho} \left( \rho \dfrac{\partial v}{\partial \rho} \right) =
	\int_0^{\infty} \dfrac{1}{\sigma(z)}\dfrac{\partial}{\partial z} \left( \sigma(z) \dfrac{\partial Z}{\partial z} \right) J_0(\lambda \rho) \lambda d \lambda =
\end{equation*}
\begin{equation*}
	= - \int_0^{\infty} Z(z,\lambda) \dfrac{1}{\rho} \dfrac{\partial}{\partial \rho} \left( \rho \dfrac{\partial J_0(\lambda \rho)}{\partial \rho} \right) \lambda d \lambda = 
	\int_0^{\infty} \lambda^2 Z(z,\lambda) J_0(\lambda \rho) \lambda d \lambda,
\end{equation*}

откуда 
\begin{equation*}
	\int_0^{\infty} \left[ \dfrac{1}{\sigma(z)} \dfrac{\partial}{\partial z} \left( \sigma(z) \dfrac{\partial Z(z,\lambda)}{\partial z} \right) - \lambda^2 Z(z,\lambda) \right] J_0(\lambda \rho) \lambda d \lambda = 0.
\end{equation*}

$\forall z > 0, \rho \geqslant 0$ Из взаимо-однозначности преобразования Ф-Б сдедует
\begin{equation}
\begin{cases}
	\dfrac{1}{\sigma(z)} \dfrac{\partial}{\partial z} \left( \sigma(z) \dfrac{\partial Z(z,\lambda)}{\partial z} \right) - \lambda^2 Z(z,\lambda) = 0;\\
	Z(\infty, \lambda) = 0;\\
	\dfrac{\partial Z(0,\lambda)}{\partial z} = -q.
\end{cases}
\label{Z system}
\end{equation}

Поскольку $\dfrac{\partial v}{\partial z} \bigg|_{z=0} = 0$, $\dfrac{q}{\sqrt{\rho^2 + z^2}} = q \int_0^{\infty} e^{-\lambda z} J_0(\lambda \rho) d \lambda$, т.е. $Z(z,\lambda) = q \dfrac{e^{-\lambda z}}{\lambda} + \overline{Z}(z, \lambda)$, $\dfrac{\partial \overline{Z}}{\partial z} \bigg|_{z=0} = 0$.

Продифференцируем (\ref{Z system}) по $z$ и введём $Z_1 = -\sigma \dfrac{\partial Z(z,\lambda}{\partial z} = 0$:
\begin{equation*}
	\sigma(z) \dfrac{\partial}{\partial z} \left( \dfrac{1}{\sigma(z)}\dfrac{\partial Z_1(z,\lambda}{\partial z} \right) - \lambda^2 Z_1(z,\lambda) =0.
\end{equation*}


Ввдём новую независимую переменную $ \dfrac{1}{\sigma(z)}\dfrac{\partial}{\partial z} = \dfrac{\partial}{\partial x}$, т.е. $\sigma(z) dz = dx \Rightarrow x(z) = \int_0^z \sigma(\xi) d\xi$. 
Пусть $\sigma = \hat{\sigma}(x(z))$. Далее будем опускать шапку над $\sigma$
\begin{equation}
\begin{cases}
	Z_1^{''} - \dfrac{\lambda}{\sigma^2(x)} Z_1(x,\lambda) = 0;\\
	Z_1(0,\lambda) = q\sigma(0);\\
	Z_1(\infty,\lambda) = 0,
\end{cases}
\label{Z1 system}
\end{equation}

где $Z_1^{''} = \dfrac{\partial^2}{\partial x^2} Z(x,\lambda)$.

Эти условия определяют $Z_1(x,\lambda)$ однозначно поскольку решение (\ref{Z1 system}s) при условиях

\begin{lemma}
	$Z_1'(x,\lambda) < 0 \forall x \geqslant 0$ и $ \forall \lambda > 0$.
\end{lemma}
%\begin{proof}
%	1) Если $Z_1'(x,\lambda) \geqslant 0$, то $Z_1'(x,\lambda) < 0 \forall x > 0$.
%	Действительно, если $x_0$ - превый нуль $Z_1'(x,\lambda)$, то $Z_1'(x,\lambda) \geqslant 0$ на $[0,x_0]$, $Z_1(0,\lambda) > 0 \Rightarrow Z_1(x,\lambda) > 0$ на $[0,x_0]$. 
%	В силу уравнения $Z_1^{''}(x,\lambda) > 0$ т.е. $Z(x,\lambda)$ ......
%\end{proof}

\begin{lemma}
	Функция Грина $G(x,\xi,\lambda)$ прямой задачи
	\begin{equation*}
	\begin{cases}
		y''(x,\lambda) - \dfrac{\lambda^2}{\sigma^2(x)} y(x,\lambda) = -f(x), & x \geqslant 0;\\
		y(0,\lambda) = y(\infty,\lambda) = 0.
	\end{cases}
	\end{equation*}
	такова что 
	\begin{enumerate}
		\item $G(x,\xi,\lambda) \geqslant 0$, $ 0 \leqslant z,\xi \leqslant 0$ , $\lambda > 0$.
		\item Если $G_i(x,\xi,\lambda)$ соответствует $\sigma_i(x)$ и $\sigma_1(x) \geqslant \sigma_2(x) \forall x \geqslant 0$, то 
		$G_1(x,\xi,\lambda) \leqslant G_2(x,\xi,\lambda) \forall x,\xi \geqslant 0, \forall \lambda > 0$.
	\end{enumerate}
\end{lemma}
\begin{proof}
	........
	где $W = y_2(x,\lambda) y_1'(x,\lambda) - y_1(x,\lambda)y_2'(x,\lambda)$, $y_1, y_2$ - линейно независимые решения
	.......
	
\end{proof}

\begin{theorem}
	Пусть теперь $\sigma_1(x) \geqslant \sigma_2(x)$ при $\forall x > 0$
\end{theorem}

\begin{hence}
	Если $\sigma_1(x) \geqslant \sigma_2(x)$ $\{\sigma_1(x) = \sigma_2(x) = \overline{\sigma_1}$ при $ x \geqslant x_1 \}$, то $Z_1(x,\lambda) \geqslant Z_2(x,\lambda)$.
\end{hence}
\begin{proof}
	\begin{equation*}
	\begin{cases}
		Z_i^{''} - \dfrac{\lambda^2}{\sigma_i^2(x)} Z_i(x,\lambda) = 0;\\
		Z_i(0,\lambda) = q\sigma(0);\\
		Z_i(\infty,\lambda) = 0.
	\end{cases}
	\end{equation*}
	
	Для разности $Z = Z_1 - Z_2$
\end{proof}

\begin{theorem}
	(Единственности)
	Если $Z(x,\lambda)$ определяется как решение уравнения 
	\begin{equation*}
	\begin{cases}
		Z^{''} - \dfrac{\lambda^2}{\sigma_i^2(x)} Z(x,\lambda) = 0;\\
		Z_i(\infty,\lambda) = 0.
	\end{cases}
	\end{equation*}
	где $\sigma(x)$ - кусочно аналитическая функция из класса $\sum = \{ \sigma(x) : \sigma(x) > 0$ при $x \in [0,x_1], \sigma = const > 0$ при $x \geqslant x_1\}$, то $\sigma(x)$ однозначно определяется значениями (импеданса):
	\begin{equation*}
		R(\lambda) = \dfrac{Z'(0,\lambda)}{Z(0,\lambda)}
	\end{equation*}
\end{theorem}
\begin{proof}
	Предположим противоположное, т.е. что существуют $\sigma_1(x) \neq \sigma_2(x)$ для 
	
	.......
	
	Рассмотрим $[x_2,x_3] \subset (0, x_0)$ такого, что $x_2 / \sigma_m < x_3/\sigma_M$, где $\sigma_m \leqslant \sigma_1(x) \leqslant \sigma_M$.
	На этом отрезке $q(x) \geqslant q_0 > 0$.
	Далее для интеграла в () имеем:
	
\end{proof}


\end{document} 