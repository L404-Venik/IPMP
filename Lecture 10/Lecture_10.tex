\documentclass{article}

\usepackage[utf8x]{inputenc}
\usepackage[english,russian]{babel}
\usepackage{graphicx}
\usepackage{amsmath}
\usepackage{amssymb}
\usepackage{extarrows}
\usepackage{vmargin}
\usepackage{MnSymbol}
\setpapersize{A4}
\setmarginsrb{2cm}{2cm}{2cm}{2cm}{0pt}{0mm}{0pt}{13mm}

%user commands
\newtheorem{theorem}{Теорема}
\newtheorem{lemma}{Лемма}

\newenvironment{proof}{\paragraph{Доказательство:}}{\hfill$\blacksquare$}
\newenvironment{definition}{ \paragraph{Определение:}}{\hfill $\blacktriangleleft$}
\newenvironment{observation}{ \paragraph{Замечание:}}{\\}
\newenvironment{hence}{ \paragraph{Следствие:}}{}


\begin{document}

\centerline{\large Курс лекция для магистров ВМК МГУ}
\centerline {\textbf{\LARGE Обратные задачи математической физики}}
\centerline {Затехал Строков Вениамин 2025}

\vspace{0.4cm}

\centerline{\LARGE 	Лекция 10. Единственность решения обратной задачи рассеяния}
\centerline{\LARGE для системы Дирака }

\vspace{0.5cm}


Вернёмся к задаче рассеяния для системы Дирака:
\begin{equation}
\begin{cases}
	v_t + v_x + z(x)u = 0, & x,t>0;\\
	u_t - u_x - z(x)v = 0;\\
	v(x,0) = u(x,0) = 0, & x > 0;\\
	v(0,t) = \varphi(t), & t \geqslant 0.
\end{cases}
\label{Dirak system}
\end{equation}

Поставим следующую обратную задачу на отрезке $[0,T]$:
\begin{enumerate}
	\item Пусть $\varphi(t) \in C^1[0,2T]$, $\varphi(0) = q \neq 0$.
	\item Пусть $f(t) \in C^1[0,2T]$, $f(0) = 0$, $f(t) = u(0,t)$.
\end{enumerate}

Найти $z(x) \in C[0,T]$.

\begin{theorem}
	(Единственности)
	Пусть выполнены условия 1) и 2) и $T>0$. Тогда обратная задача не может иметь более едного решения $z(x) \in C[0,T]$. Эти условия также гарантируют разрешимость в малом.
\end{theorem}
\begin{proof}
	(от противного)
	
	Пусть $\exists z_1 \neq z_2$. Проинтегрируем (\ref{Dirak system}) вдоль характеристики $x + \tau = 2t$:
	
%\includegraphics[scale=•]{•}

\begin{equation*}
	u_j(0,2t) - u_j(x,x) = \int_0^t v_j(x,2t-x) z_j(x) dx,
\end{equation*}

откуда
\begin{equation*}
	\int_0^t v_j(x,2t-x)z_j(x) dx = f_j(2t).
\end{equation*}

Продифференцируем это равенство, представив $v$ как $v_j(x,t) = \varphi(t-x) + V_j(x,t)$:
\begin{equation*}
	qz_j(t) + 2 \int_0^t V_{j,t}(x,2t-x) z_j(x) dx = 2f_j'(2t), \quad 0 \leqslant t \leqslant T.
\end{equation*}

Для $z_1-z_2 = z$ имеем:
\begin{equation*}
	qz(t) + \int_0^t V_{1,t} (x,2t-x) z_1(x) dx + \int_0^t V_{2,t}(x,2t-x) z_2(x) dx = 0.
\end{equation*}

Используя оценку устойчивости для $V_t$, получаем для любого $0 < t \leqslant T$:
\begin{equation*}
	||z||_{C[0,T]} \leqslant \text{const} * \int_0^t ||z||_{c[0,\tau]} d\tau.
\end{equation*}

Из Леммы Гронуолла $\Rightarrow z = 0$ на $[0,T]$.
\end{proof}

\subsection*{Обратная кинематическая задача сейсмики}

%\includegraphics[scale=•]{•}

ОЗ: по времени $T$ первого вступления в точку дневной поверхности $X$ найти $v(z) $ - скорость.

Далее считаем что $v(0) = 1$, $v(z)$ монотонно возрастает и $v \in C^1[0,\infty)$.
Время распространения сигнала определяется уравнением эйконала:
\begin{equation*}
	t_x^2 + t_z^2 = h^2(z), \quad h(z) = \dfrac{1}{v(z)}, \quad z \geqslant 0, -\infty < x < \infty.
\end{equation*}

Это уравнение описывает фронты волн.
Сопоставим ему систему Коши, записав в каноническом виде $p^2 + q^2 - h^2(z) = 0$.
\begin{equation}
	\dfrac{dx}{ds} = 2p, \quad 
	\dfrac{dz}{ds} = 2q, \quad
	\dfrac{dt}{ds} = 2(p^2 + q^2) = 2h^2(z), \quad
	\dfrac{dp}{ds} = 0, \quad
	\dfrac{dq}{ds} = 2h'(z) h(z).
	\label{Cauchy problem}
\end{equation}

с начальными условиями при $s=0$:
\begin{equation*}
	x(0) = z(0) = t(0) = 0; \quad
	p(0) = p; \quad
	q(0) = \sqrt{1 - p^2}.
\end{equation*}

Из (\ref{Cauchy problem}) следует 
$p =$ const,
$ q^2(z) = h^2(z) - p^2$,
$\dfrac{dt}{dz} = \pm \dfrac{h^2(z)}{\sqrt{h^2(z) - p^2}}$,
$\dfrac{dx}{dz} = \pm \dfrac{p}{\sqrt{h^2(z) - p^2}}$ 
для таких $z: h(z) \geqslant p$. 
\begin{equation}
	t(p) = 2 \int_0^{z_m(p)} \dfrac{h^2(z) dz}{\sqrt{h^2(z) - p^2}}, \quad
	x(p) = 2 \int_0^{z_m(p)} \dfrac{p dz}{\sqrt{h^2(z) - p^2}},
	\label{parametrization}
\end{equation}

где $z_m(p)$ определяется равенством $h(z_m(p)) = p$, а при $z < z_m(p)$ $h(z) > p$.

Нам известна функция $T(x)$ - годограф первых вступлений. Вопрос, как его параметризовать?

Докажем, что $dt(p) = p dx(p)$, а тем самым $\dfrac{dT}{dX} = p$, когда $T = t(p)$, $X = x(p)$. Введём функцию
\begin{equation*}
	\tau(p) 2 \int_0^{z_m(p)} \sqrt{h^2(z) - p^2} dz = t(p) - px(p).
\end{equation*}
\begin{equation*}
	\tau'(p) = -2 \int_0^{z_m(p)} \dfrac{p dz}{\sqrt{h^2(z) - p^2}} = x(p) =  t'(p) - px'(p) - x(p) \Rightarrow
\end{equation*}
\begin{equation*}
	t'(p) = p x'(p) \quad \Rightarrow \quad
	dT = t'(p)dp, \quad dX = x'(p) dp, \quad
	\dfrac{dT(x)}{dX} \bigg|_{T = t(x), X = x(p)} = p,
\end{equation*}

таким образом, годограф параметризован: $T(x(p)) = t(p)$, $X = x(p)$.

Введём функцию $Z(p) = \inf\{z | h(z) \leqslant p\}$, т.е. $Z(p)$ - наименьшая глубина, куда не проникает с сейсмический луч с параметром $p$.

Запишем (\ref{parametrization}) в виде
\begin{equation}
	t(p) = 2 \int_0^{p} \dfrac{h^2 }{\sqrt{h^2 - p^2}} dZ(h), \quad
	x(p) = 2 \int_0^{p} \dfrac{p }{\sqrt{h^2 - p^2}} dZ(h),
	\label{parametrization2}
\end{equation}

т.к. $h(0) = 1$.

Домножим (\ref{parametrization2}) на $\dfrac{1}{p^2 - q^2}$ и проинтегрируем по $p$ от $q$ до 1 (это возможно в силу теорему Фубини) $\Rightarrow$ 
\begin{equation*}
	\int_q^1 \dfrac{x(p) dp}{\sqrt{p^2 - q^2}} =
	 -2 \int_q^1 \dfrac{1}{\sqrt{p^2 - q^2}} \int_p^1 \dfrac{p dZ(h)}{\sqrt{h^2 - p^2}} dp =
	  \iint_D dp dh =
\end{equation*}
\begin{equation*}
	  = -2 \int_q^1 \int_q^h \dfrac{p dp}{\sqrt{(p^2 - q^2) (h^2 - p^2)}} dZ(h) = \pi Z(q)
	  \quad \Rightarrow 
\end{equation*}
\begin{equation*}
	Z(q) = \dfrac{1}{\pi} \int_q^1 \dfrac{x(p) dp}{\sqrt{p^2-q^2}}
\end{equation*}

эта формула доказывает единственность решения ОЗ. $h(z)$ - функция обратная к $Z(h)$, поскольку $h(Z(h))=h$.

\end{document} 