\documentclass{article}

\usepackage[utf8x]{inputenc}
\usepackage[english,russian]{babel}
\usepackage{graphicx}
\usepackage{amsmath}
\usepackage{amssymb}
\usepackage{extarrows}
\usepackage{vmargin}
\usepackage{MnSymbol}
\setpapersize{A4}
\setmarginsrb{2cm}{2cm}{2cm}{2cm}{0pt}{0mm}{0pt}{13mm}
%\usepackage{cmap}


\begin{document}

\centerline{\large Курс лекция для магистров ВМК МГУ}
\centerline {\textbf{\LARGE Обратные задачи математической физики}}
\centerline {Затехал Строков Вениамин 2025}

\vspace{0.4cm}

\centerline{\LARGE Лекция 2. Обратные задачи теории потенциала}

\vspace{1cm}
\centerline{\large Задача продолжения потенциала}

\begin{align}
\begin{cases}
\Delta u = 0, & 0 < x < a, 0 < y < b;\\
u(0,y) = u(a,y) = 0, & 0 \leqslant y \leqslant b;\\
u(x,0) = \varphi(x), u_y(x,0) = \psi(x), & 0 \leqslant x \leqslant a.
\end{cases}
\end{align}

Эту задачу можно рассматривать как обратную к задаче Дирихле, положив $u(x,d) = h(x)$, $0 \leqslant x \leqslant a$, где $d \in (0,b]$ и считая $\psi(x)$ пока неизвестной. 

Запишем решение задачи Дирихле для уравнения Лапласа в $\Pi = [0,a] \times [o,d]$:
\[
u(x,y) = 
\dfrac{2}{a} \sum_{n=1}^{\infty} \int_0^a \varphi(\xi) \sin(\dfrac{ \pi n \xi }{a}) d \xi *
\dfrac{\sh (\dfrac{\pi n}{a} (d-y))}{\sh( \dfrac{\pi n}{a} d)} \sin \dfrac{\pi n}{a} x +
\dfrac{2}{a} \sum_{n=1}^{\infty} \int_0^a h(\xi) \sin(\dfrac{ \pi n \xi }{a}) d \xi *
\dfrac{\sh (\dfrac{\pi n}{a} y)}{\sh (\dfrac{\pi n}{a} d)} \sin \dfrac{\pi n}{a} x
\]

\[
u_y(x,0) = 
- \sum_{n=1}^{\infty} \dfrac{2 \pi n}{a^2} \int_0^a \varphi(\xi) \sin(\dfrac{ \pi n \xi }{a}) d \xi*
\cth \dfrac{\pi n d}{a} \sin \dfrac{\pi n}{a} x +
\sum_{n=1}^{\infty} \dfrac{2 \pi n}{a^2} \int_0^a h(\xi) \sin(\dfrac{ \pi n \xi }{a}) d \xi *
\dfrac{\sin( \dfrac{\pi n}{a} x)}{\sh (\dfrac{\pi n}{a} d)} 
\]

Если




\end{document}