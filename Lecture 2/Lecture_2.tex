\documentclass{article}

\usepackage[utf8x]{inputenc}
\usepackage[english,russian]{babel}
\usepackage{graphicx}
\usepackage{amsmath}
\usepackage{amssymb}
\usepackage{extarrows}
\usepackage{vmargin}
\usepackage{MnSymbol}
\setpapersize{A4}
\setmarginsrb{2cm}{2cm}{2cm}{2cm}{0pt}{0mm}{0pt}{13mm}
%\usepackage{cmap}


\begin{document}

\centerline{\large Курс лекция для магистров ВМК МГУ}
\centerline {\textbf{\LARGE Обратные задачи математической физики}}
\centerline {Затехал Строков Вениамин 2025}

\vspace{0.4cm}

\centerline{\LARGE Лекция 2. Обратные задачи теории потенциала}

\vspace{1cm}
\centerline{\large Задача продолжения потенциала}

\begin{align}
\begin{cases}
\Delta u = 0, & 0 < x < a, 0 < y < b;\\
u(0,y) = u(a,y) = 0, & 0 \leqslant y \leqslant b;\\
u(x,0) = \varphi(x), u_y(x,0) = \psi(x), & 0 \leqslant x \leqslant a.
\end{cases}
\end{align}

Эту задачу можно рассматривать как обратную к задаче Дирихле, положив $u(x,d) = h(x)$, $0 \leqslant x \leqslant a$, где $d \in (0,b]$ и считая $\psi(x)$ пока неизвестной. 

Запишем решение задачи Дирихле для уравнения Лапласа в $\Pi = [0,a] \times [o,d]$:
\[
u(x,y) = 
\dfrac{2}{a} \sum_{n=1}^{\infty} \int_0^a \varphi(\xi) \sin(\dfrac{ \pi n \xi }{a}) d \xi *
\dfrac{\sh (\dfrac{\pi n (d-y)}{a})}{\sh( \dfrac{\pi n d}{a})} \sin (\dfrac{\pi n x}{a}) +
\dfrac{2}{a} \sum_{n=1}^{\infty} \int_0^a h(\xi) \sin(\dfrac{ \pi n \xi }{a}) d \xi *
\dfrac{\sh (\dfrac{\pi n y}{a})}{\sh (\dfrac{\pi n d}{a})} \sin( \dfrac{\pi n x}{a})
\]

\[ %написал как в лекции, но кажется это неправильно
u_y(x,0) = 
- \sum_{n=1}^{\infty} \dfrac{2 \pi n}{a^2} \int_0^a \varphi(\xi) \sin(\dfrac{ \pi n \xi }{a}) d \xi*
\cth( \dfrac{\pi n d}{a} )\sin( \dfrac{\pi n x}{a} )+
\sum_{n=1}^{\infty} \dfrac{2 \pi n}{a^2} \int_0^a h(\xi) \sin(\dfrac{ \pi n \xi }{a}) d \xi *
\dfrac{\sin( \dfrac{\pi n x}{a})}{\sh (\dfrac{\pi n d}{a})} 
\]

Если $\varphi(x)$ и $\psi(x)$ известны, то обратная задача сводится к задаче определения $h(x)$ из уравнения

\[
\sum_{n=1}^{\infty} n \int_0^a h(\xi) \sin(\dfrac{ \pi n \xi }{a}) d \xi *
\dfrac{\sin( \dfrac{\pi n x}{a})}{\sh (\dfrac{\pi n d}{a})} = g(x), \hspace{1cm} 0 \leqslant x \leqslant a;
\]
где 
\[
g(x) = \psi(x) \frac{a^2}{2 \pi} + \sum_{n=1}^{\infty} n \int_0^a  \varphi(\xi) \sin(\dfrac{ \pi n \xi }{a}) d \xi* \cth (\dfrac{\pi n d}{a}) \sin (\dfrac{\pi n x}{a}).
\]

Единственность $h(x)$ следует из полноты системы $\{\sin (\dfrac{\pi n x}{a})\}_{n=1}^{\infty}$ на $[0,a]$ в $L_2$.


\[
\int_0^a G(x,\xi) h(\xi) d \xi = g(x), \hspace{1cm} 0 \leqslant x \leqslant a,
\]
\[
G(x,\xi) =  \sum_{n=1}^{\infty} n \sin (\dfrac{\pi n x}{a})  \dfrac{\sin (\dfrac{\pi n \xi}{a})}{\sh (\dfrac{\pi n d}{a})}.
\]

Ядро $G(x,\xi)$ непрерывно и дифференцируемо в $[0,a]^2$. Таким образом задача решения этого интегрального уравнения некорректна как из $L_2[0,a]$ в $L_2[0,a]$, так и из $C[0,a]$ в $C[0,a]$.

Получим оценку условной устойчивости. Пусть $u$, $u_{xx}$, $u_{xy}$, $u_{yy}$, непрерывны в $\Pi = [0,a] \times [0,b]$ и $\Delta u = 0$ в $\Pi$, $u \nequiv 0$ в $\Pi$.
Рассмотрим функцию

\[
p(y) = \int_0^a u^2 (x,y) dx,
\]
\[
p'(y) = 2 \int_0^a u (x,y) u_y(x,y) dx,
\]

\begin{multline*}
p''(y) = 2 \int_0^a u (x,y) u_{yy}(x,y) dx + 2 \int_0^a u_y^2 (x,y) dx = \{ u_{yy} = - u_{xx} \} = \\
= -2 \int_0^a u (x,y) u_{xx}(x,y)dx + 2 \int_0^a u_y^2(x,y)dx 
=  2 \int_0^a (u_x^2(x,y) + u_y^2(x,y))dx.
\end{multline*}

Заметим, что

\begin{multline*}
\dfrac{d}{dy} \int_0^a u_x^2(x,y)dx =
 2 \int_0^2 u_x(x,y) u_{xy}(x,y) dx = 
 2 u_x(x,y) u_y(x,y) \bigg|_0^a - 2 \int_0^a u_{xx}(x,y) u_y(x,y) dx =\\
= \{ u_y(x,y) = 0, u_{xx}(x,y) = -u_{yy}(x,y) \} =
 2 \int_0^a u_{yy}(x,y) u_y(x,y) dx = \dfrac{d}{dy} \int_0^a u_y^2(x,y)dx \Rightarrow
\end{multline*}

\[
\int_0^a u_x^2(x,y)dx = \int_0^a u_y^2(x,y) dx + \int_0^a (u_x^2(x,0) - u_y^2(x,y)) dx \Rightarrow
\]

\[
p''(y) = 2 \int_0^a u(x,y) u_{yy}(x,y)dx + 2 \int_0^a u_y^2(x,y) dx = 4 \int_0^a u_y^2(x,y) dx + 4q,
\]
\[
q = \dfrac{1}{2} \int_0^a (u_x^2(x,0) - u_y^2(x,0))dx.
\]

Рассмотрим функцию $k(y) = \ln (p(y) + |q|)$.
\[
k'(y) = \dfrac{p'(y)}{p(y) + |q|}, 
\]
\[
k''(y) = \dfrac{p''(y) (p(y) + |q|) - (p'(y))^2}{(p(y) + |q|)^2} = 
\]
\[
=\dfrac{4}{(p(y) + |q|)^2} \bigg[ \int_0^a u_y^2(x,y) dx \int_0^a u^2(x,y)dx + q \int_0^a u^2(x,y) dx + |q| \int_0^a u_y^2(x,y)dx + q |q| - (\int_0^a u(x,y) u _y(x,y)dx)^2 \bigg]
\]

$\Rightarrow $ по неравенству Коши-Буняковского

\[
k''(y) \geqslant \dfrac{4q}{(p(y) + |q|)} \geqslant -4.
\]

Рассмотрим функцию $\omega(y) = k(y) + 2y(y-b)$. Т.к. $\omega '' (y) = k''(y) + 4 \geqslant 0 $ при $y \in [a,b]$, следует

\[
\omega(y) \leqslant \omega(0)\dfrac{(b-y)}{b} + \omega(b) \dfrac{y}{b} \Rightarrow
\]
\[
k(y) \leqslant k(0)\dfrac{(b-y)}{b} + k(b)\dfrac{y}{b} - 2y(y-b) \Rightarrow
\]
\[
\ln (p(y) +|q|) \leqslant \dfrac{b-y}{b} \ln (p(0) + |q|) + \frac{y}{b} \ln(p(b) +|q|) - 2y(y-b) \Rightarrow
\]
\[
p(y) \leqslant (p(0) + |q|)^{\frac{b-y}{b}} * (p(b) +|q|)^{\frac{y}{b}} e^{2y(b-y)} - |q|
\]

Получим оценку устойчивости.
Пусть $u_j(x,y)$ удовлетворяет краевым условиям $\varphi_j(x)$ и $\psi_j(x)$ и всем условиям поставленным для $u(x,y)$ ране, и пусть

\[
\int_0^a u_j^2(x,b)dx \leqslant C^2.
\]
Положим $u(x,y) = u_1(x,y) - u_2(x,y)$, тогда  $|| u(x,b)||_{L_2[0,a]}^2 \leqslant 4 C^2$

Из полученного для $p(y)$ неравенства имеем
\[
||u_1(x,y) - u_2(x,y) ||_{L_2[0,a]}^2 \leqslant (|| \varphi_1(x) - \varphi_2(x)||_{L_2[0,a]}^2 + |q|) ^{\frac{b-y}{b}} (4C^2 +|q|)^{\frac{y}{b}} e^{2y(b-y)} - |q|,
\]
где
\[
q = \dfrac{1}{2} \int_0^a [(\varphi_1'(x) - \varphi_2'(x))^2 + (\psi_1(x) - \psi_2(x))^2 ] dx,
\]
\[
|q| \leqslant \dfrac{1}{2} ||\varphi_1' - \varphi_2'||_{L_2[0,a]}^2 + \dfrac{1}{2} ||\psi_1 - \psi_2||_{L_2[0,a]}^2.
\]


\vspace{1cm}
\newpage
\centerline{\large Обратная задача гравиметрии}

ОЗ состоит в определении формы и плотности тела по ньютоновскому потенциалу. Рассмотрим ньютонов потенциал тела $T$ с протностью $\rho(M)$:
\[
u(M) = \iiint_T \dfrac{\rho(M')}{z_{MM'}}d\tau_{M'}
\]
Известно, что плотность $\rho(M)$ по потенциалу, заданному на сфере $S_R$ (поверхности тела $T$) не определяется однозначно.

Пусть $\rho_1 \neq \rho_2$ и создают одинаковый потенциал для тела $T$. Обозначим $\rho_1 - \rho_2 = \sigma(M)$, и $u_1|_{S_R} = u_2|_{S_R} \Rightarrow u_1 = u_2$ вне сферы $S_R$ и, более того вплоть до границы тела T.

Рассмотрим $\sigma(M) = \Delta \omega(M)$, где $ \omega \in C^2(T)$ и $\omega|_{\partial T} = \dfrac{\partial \omega}{\partial n} \bigg|_{\partial T} = 0$. Тогда из II формулы Грина имеем для любой гармонической в $T$ функции $v$: 
\[
\iiint_T (v \Delta \omega - \omega \Delta v) d\tau = \iint_{\partial T} (v \dfrac{\partial \omega}{\partial n} - \omega \dfrac{\partial v}{\partial n} ) ds = 0, 
\]
т.е.
\[
\iiint_T \sigma(M') v(M') d \tau_{M'} = 0.
\]
откуда следует при $v(M') = \dfrac{1}{z_{MM'}}, M \nin T \Rightarrow$
\[
u_1(M') - u_2(M')= \iiint_T \dfrac{\sigma(M')}{z_{MM'}}d\tau_{M'} = 0.
\]
Таким образом, если $\sigma(M) = \rho_1 - \rho_2$, то внешний потенциал $u(M) = 0$ вне $T$.

\noindent Покажем, что справедливо и обратное.
Пусть 
$$
u(M) = - \dfrac{1}{4 \pi} \iiint_T \dfrac{\sigma(M')}{z_{MM'}} d \tau_{M'} = 0, \hspace{1cm} M \nin T.
$$
тогда $u(P)|_{P \in \partial T} = \dfrac{\partial u(P)}{\partial n} \bigg|_{P \in \partial T} = 0$, т.е. вплоть до границы, а $u(M) \in C^2(T)$  удовлетворяет уравнению Пуассона:
\[
\Delta u = \sigma (M), \hspace{1cm}  M \in T
\]

Таким образом, описанным классом $\sigma(M)$ исчерпывается всё множество неединственности для плотности при фиксированной форме тела $T$.




\end{document}