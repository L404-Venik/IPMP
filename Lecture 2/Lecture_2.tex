\documentclass{article}

\usepackage[utf8x]{inputenc}
\usepackage[english,russian]{babel}
\usepackage{graphicx}
\usepackage{amsmath}
\usepackage{amssymb}
\usepackage{extarrows}
\usepackage{vmargin}
\usepackage{MnSymbol}
\setpapersize{A4}
\setmarginsrb{2cm}{2cm}{2cm}{2cm}{0pt}{0mm}{0pt}{13mm}
%\usepackage{cmap}


\begin{document}

\centerline{\large Курс лекция для магистров ВМК МГУ}
\centerline {\textbf{\LARGE Обратные задачи математической физики}}
\centerline {Затехал Строков Вениамин 2025}

\vspace{0.4cm}

\centerline{\LARGE Лекция 2. Обратные задачи теории потенциала}

\vspace{1cm}
\centerline{\large Задача продолжения потенциала}

\begin{align}
\begin{cases}
\Delta u = 0, & 0 < x < a, 0 < y < b;\\
u(0,y) = u(a,y) = 0, & 0 \leqslant y \leqslant b;\\
u(x,0) = \varphi(x), u_y(x,0) = \psi(x), & 0 \leqslant x \leqslant a.
\end{cases}
\end{align}

Эту задачу можно рассматривать как обратную к задаче Дирихле, положив $u(x,d) = h(x)$, $0 \leqslant x \leqslant a$, где $d \in (0,b]$ и считая $\psi(x)$ пока неизвестной. 

Запишем решение задачи Дирихле для уравнения Лапласа в $\Pi = [0,a] \times [o,d]$:
\[
u(x,y) = 
\dfrac{2}{a} \sum_{n=1}^{\infty} \int_0^a \varphi(\xi) \sin(\dfrac{ \pi n \xi }{a}) d \xi *
\dfrac{\sh (\dfrac{\pi n (d-y)}{a})}{\sh( \dfrac{\pi n d}{a})} \sin (\dfrac{\pi n x}{a}) +
\dfrac{2}{a} \sum_{n=1}^{\infty} \int_0^a h(\xi) \sin(\dfrac{ \pi n \xi }{a}) d \xi *
\dfrac{\sh (\dfrac{\pi n y}{a})}{\sh (\dfrac{\pi n d}{a})} \sin( \dfrac{\pi n x}{a})
\]

\[
u_y(x,0) = 
- \sum_{n=1}^{\infty} \dfrac{2 \pi n}{a^2} \int_0^a \varphi(\xi) \sin(\dfrac{ \pi n \xi }{a}) d \xi*
\cth( \dfrac{\pi n d}{a} )\sin( \dfrac{\pi n x}{a} )+
\sum_{n=1}^{\infty} \dfrac{2 \pi n}{a^2} \int_0^a h(\xi) \sin(\dfrac{ \pi n \xi }{a}) d \xi *
\dfrac{\sin( \dfrac{\pi n x}{a})}{\sh (\dfrac{\pi n d}{a})} 
\]

Если $\varphi(x)$ и $\psi(x)$ известны, то обратная задача сводится к задаче определения $h(x)$ из уравнения

\[
\sum_{n=1}^{\infty} n \int_0^a h(\xi) \sin(\dfrac{ \pi n \xi }{a}) d \xi *
\dfrac{\sin( \dfrac{\pi n x}{a})}{\sh (\dfrac{\pi n d}{a})} = g(x), \hspace{1cm} 0 \leqslant x \leqslant a;
\]
где 
\[
g(x) = \psi(x) \frac{a^2}{2 \pi} + \sum_{n=1}^{\infty} n \int_0^a  \varphi(\xi) \sin(\dfrac{ \pi n \xi }{a}) d \xi* \cth (\dfrac{\pi n d}{a}) \sin (\dfrac{\pi n x}{a}).
\]

Единственность $h(x)$ следует из полноты системы $\{\sin (\dfrac{\pi n x}{a})\}_{n=1}^{\infty}$ на $[0,a]$ в $L_2$.


\[
\int_0^a G(x,\xi) h(\xi) d \xi = g(x), \hspace{1cm} 0 \leqslant x \leqslant a,
\]
\[
G(x,\xi) =  \sum_{n=1}^{\infty} n \sin (\dfrac{\pi n x}{a})  \dfrac{\sin (\dfrac{\pi n \xi}{a})}{\sh (\dfrac{\pi n d}{a})}.
\]

Ядро $G(x,\xi)$ непрерывно и дифференцируемо в $[0,a]^2$. Т.о. задача решения этого интегрального уравнения некорректна как из $L_2[0,a]$ в $L_2[0,a]$, так и из $C[0,a]$ в $C[0,a]$.

Получим оценку условной устойчивости. Пусть $u$, $u_{xx}$, $u_{xy}$, $u_{yy}$, непрерывны в $\Pi = [0,a] \times [0,b]$ и $\Delta u = 0$ в $\Pi$, $u \ne 0$ в $\Pi$.

Рассмотрим функцию

\[
P(y) = \int_0^a u^2 (x,y) dx,
\]
\[
P'(y) = 2 \int_0^a u (x,y) u_y(x,y) dx,
\]

\begin{multline*}
P''(y) = 2 \int_0^a u (x,y) u_{yy}(x,y) dx + 2 \int_0^a u_y^2 (x,y) dx = \\
= -2 \int_0^a u (x,y) u_{xx}(x,y)dx + 2 \int_0^a u_y^2(x,y)dx = \\
=  2 \int_0^a (u_x^2(x,y) + u_y^2(x,y))dx, 
\end{multline*}

заметим, что



\end{document}