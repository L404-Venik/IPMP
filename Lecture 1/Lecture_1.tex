\documentclass{article}

\usepackage[utf8x]{inputenc}
\usepackage[english,russian]{babel}
\usepackage{amsmath}
\usepackage{amssymb}
%\usepackage{cmap}


\begin{document}


\centerline{\sc Лекция 1. Математические модели естествознания и обраные задачи}

Математическую физику ъарактеризует то, что на основе установленных закономерностей возможна постановка задач, замкнуто описывающих исследуемый процесс или явление.

В рамках математической физики осрновным объектом изучения являются причинно-следственные связи и их последствия.

$$ Z \rightarrow {A} \rightarrow U $$

Причина Z, закономерность A, следисвие U - это математическая модель прямой задачи (ПЗ, PD)

Прямые задачи характеризуют (как правило):\\
1) однозначность резултата,\\
2) непрерывная зависимость U от Z;\\
при задании достаточного объёма дополнительных условй, например, начальных, краевых и т.п. Это приводит к возможности детерминированного описания мира (концепция детерминизма).

Рассмотрим теперь класс задач, в которых по следствию определяется причина:

$$ U \rightarrow {R} \rightarrow Z $$

Следствие U, формальня связь R, причина Z - это математическая связь ообратной задачи (ОЗ).

Некоторые свойства ОЗ:\\
1) ОЗ физически не реализуемы;\\
2) ОЗ возникают при исследовании объектов и явлений недоступных непосмредственному изучения (геофизика, астрономия, радиолокация, неразрушающая дигностика в медицине, технике, физика микромира, и т.д.)\\
3) отсутствие устойчивости, т.е. нет пепрерывной зависимости от данных наблюдений, измерений и т.п.

Пример 1. ОЗ рассеяния для стержня

???????

ОЗ: найти неоднородность в стержне\\
$\phi (t)$ - импульсы воздействия при $x = 0$\\
$f(t)$ - измеряемые колебангия при $x = 0$\\
$\rho(x)$ - неизвестная плотность

Т.е. ОЗ рассеяния - найти $\rho(x)$

Пример 2: Обратная динамическая задача сейсмики

????????????????????????????

$z = \phi_j(x)$ - уравнения границ\\
$\rho_j(x)$ - плотности в слое

ОЗ: по рассеянному полю восстановить среду - т.е определить геологический разрез.

Пример 3. Обратная спектральная задача

???????????????????????????????????

Имеется струна переменного сечения $S(x)$ и плотности $\rho(x)$, T = const

$$ \rho(x) S(x) u_{tt} = T u_{xx}, 0 < x < l; $$

Известны частота собственных колебаний ${\lambda_k}$ и ${\phi'(0,\lambda_k)}$ при $|| \phi(x,\lambda_k)||_{L_2[0,l]}=1$, можно ли определить $\rho(x), S(X)$?

Пример 4. Обратная задача гравиметрии

???????????????????????????????????

$g(x)$ - амплитуда поля тяготения. 

Можно ли определить рудные тела по $g(x)$


Обратная задача теории потенциала

???????????????????????????????????

рудное тело или нефтяная залежь.

Можно ли найти конкретную границу, т.е. форму резервуара или рудного тела.


\centerline{\sc Корректная и некорректная задачи}

Итак ОЗ $R(U) = z$, $u$ дано, требуется найти $z$.

Пусть $u \in U$, $z \in Z$, $U,Z$ - некоторые множества, далее метрические.

Опр. Задача $R(U) = z$ называется корректной по Адамару на $U,Z$ если\\
1) $\forall u \in U \exists z = R(U), z \in Z$,\\
2) решение $z$ единственно,\\
3) $z$ непрерывно зависит от $u$.

Пример 1. Определениен правой части ОДУ.

Пусть $x \in X = C[0,T], f \in F = C[0,T]$.
$$
\begin{cases} 
x''(t) = f(t), 0 \leqslant t \leqslant T;\\
x(0) = x(0) = 0.
\end{cases}
$$ 
Задача определения $f(t)$ по $x(t)$ явдяется некорректой.

Решение существует не для $\forall x \in X$,\\
Единственность присутствует,\\
Устойчивость отсутствует:\\
$x_n = \frac{1}{n} sin(n^2t) \rightarrow 0, n \rightarrow \infty$,
$f_n$ - расходится при $n \rightarrow \infty $

Задача становится корректной при $X = C^2[0,T]$.

Замечание 1. Если отображение $R$ - линейно, то достаточно исследовать есдинственность и устойчивость нулевого решения.

Замечание 1. Задача $R(x) = f$ можно сопоставить эквивалентную задачу решения операторного уравнения $Af = x$, где

$$
Az = \int_0^t (t - \tau) f(\tau) d \tau = x(t), 0 \leqslant t \leqslant T.
$$

Пример 2. Интегральное уравнение Фредгольма I рода:

$$
Az = \int_a^b K(x,s) z(s) ds = u(x), x \in [c,d],
$$
При $K \in C([c,d] \times [a,b])$ эта задача некорректная.

Докажем для на отрезке $[0,\pi]$:

Пусть $Az_0 = u_0$ и $z_n^{(s)} = z_0^{(s)} N \sin (ns)$.

$$
u_n(x) - u_o(x) = N \int_0^{\pi} K(x,s) sin(ns) ds = N K_n(x), c \leqslant x \leqslant d,  
$$
где $K_n(x)$ - коэффициенты Фурье $K(x)$ по переменным s.

Покажем, что $K_n \rightrightarrows 0$ при $n \rightarrow \infty $.

1) Нетрудно проверить, что $K_n(x) \in C[c,d]$ и $|K_n(x)| < M, \forall n \in \mathbb{N}$.\\

2) Последотвательность $K_n(x)$ равностепенно непрерывна, поскольку $K_n(x)$ непрерывна $\Rightarrow$ равномерно непрерывна на $[c,d] \times [a,b]$, 
т.е. существует $ \omega(h)$ : 
$$
|K(x',s') - K(x'',s'')| \leqslant \omega ((x' - x'') + (s' - s'')), \omega (h) \rightarrow 0.
$$ 
Тогда 
$$
|K_n(x') - K_n(x'')| \leqslant \int_0^{\pi} |K(x',s) - K(x'',s)| sin (ns) ds \leqslant \pi \omega |x' - x''|,
$$
$w(h)$ - модуль непрерывности.

По теореме Арцела $\exists K_{n_p}(x) \rightrightarrows 0$ при $p \rightarrow \infty $ $\Rightarrow$ $K_n(x) \rightrightarrows 0$ при $n \rightarrow \infty $.
 Т.о. $\exists \varepsilon(n) > 0, \varepsilon (n) \rightarrow 0, n \rightarrow \infty $: $|K_n(x)| \leqslant \varepsilon(n)$

Выберем теперь $N = N(n)$ так, что $N(n) \rightarrow 0$ и $N(n)\varepsilon(n) \rightarrow 0$ при $n \rightarrow \infty $ ( например $N(n) = \frac{1}{\sqrt{\varepsilon (n)}}$)

Тогда $|z_n(s) - z_0(s)| = \frac{1}{\sqrt{\varepsilon (n)}} \sin(ns)$ и $max_{[0,\pi]}[z_n(s) - z_0(s)| = \frac{1}{\sqrt{\varepsilon (n)}}$, но $|u_n(x) - u_0(x)| \leqslant \sqrt{\varepsilon (n)}$, решение неустойчиво, т.е. нет неприрывной зависимости решения от правой части.

Пример 3. (Адамара)

Найти $z(x,y)$ : 
$$
z_{xx} + z_{yy} = 0, x \in \mathbb{R}, 0 < y \leqslant b,
$$$$
z(x,0)=0, z_y(x,0) = u(x), x \in \mathbb{R}.
$$

Введём для $z(x,y)$ и $u(x)$ равномерные метрики. Тогда задача $z = R(u)$ - некорректна. Действительно, для $ u_n(x) = \frac{1}{n} sin(nx) \rightarrow 0, n \rightarrow \infty $ существует единственное решение.

$$
z_n(x,y) = \frac{1}{n^2} \sin(nx) \sh(ny)
$$ 
например, при $x_n = \frac{\pi}{2n}, y =1$ при $n \rightarrow \infty $, $z_n(x_n, 1) \rightarrow \infty $


\end{document}